\section{awk}

\textit{Evolved out of grep, sed; both of which evolved out of line editor }\textbf{ed}\textit{.} 


%%%%%%%%%%%%%%%%%%%%%%%%%%%%%%%%%%%%%%%
\subsection*{Invocation}

\code{awk `<instruction>*' files} \\
\code{awk -f <script> <files>} \\
\code{awk `<instruction>*' -}\\


\textit{The most common options are:}\\
\begin{tabular}{l l}
    -f & script \ul{f}ilename to follow \\
    -F & $\Delta$ed \ul{F}ield separator to follow \\
    -v & \ul{v}ariable assignment follows \\
\end{tabular}


\textit{Example invocations:}\\
\entry{37mm}{awk `\{ print \}' in.file}{print each line}\\
\entry{37mm}{awk `\{ print \$1 \}' my.txt}{print all 1st fields}\\
\entry{45mm}{awk -f myscript high=99 file}{setting vars}\\
\entry{37mm}{awk -f script -f /dir/lib}{link a lib (2nd -f)}\\
\entry{37mm}{shell> cat myscript}{script is all awk}\\
\entry{37mm}{shell> \#!/usr/bin/awk \dots }{direct \say{shebang}}\\
\entry{37mm}{shell> myscript.sh file.in}{invoke; pass args}\\
\entry{37mm}{awk `\dots' \$*}{pass script params}\\

\textit{Input from pipe using non-default separator:}\\
\code{ls | awk \hl{-F``\textbackslash t''} `\{ print NR ``: '' \$1 \}'}\\



%%%%%%%%%%%%%%%%%%%%%%%%%%%%%%%%%%%%%%%
\subsection*{Main Input Loop}
\textit{Processing proceeds in a \say{main input loop,} which iterates over all lines in all input files, running the script against each line. Each script \say{instruction} is comprised of a pattern and block of actions:}\\

\code{<pattern> \{ <action>* \}}\\

\textit{Patterns can be regex or line numbers, eg:}\\
\entry{40mm}{\hl{/\textasciicircum \$/} \{ print ``blank line!'' \}}{match regex}\\
\entry{35mm}{\hl{NR == 1} \{ print \$0 \}}{conditional pattern}\\
\entry{35mm}{\hl{BEGIN} \{ FS=``,'' \}}{special section}\\
\entry{35mm}{\hl{\$5 \textasciitilde /TX/} \{ print \dots \}}{regex matches fld?}\\
\entry{35mm}{\hl{\$5 \textasciitilde myvar} \{ print \dots \}}{ibid, using var}\\
\entry{35mm}{\hl{\$5 !\textasciitilde /MA/} \{ print \dots \}}{ibid, negated}\\

\textit{Each record is split into fields. While the main loop processes, variables }\textbf{NR}, \textbf{NF}, \textit{respectively contain index \ul{n}umber of the current \ul{r}ecord, and total \ul{n}umber of \ul{f}ields within this record. Eg, to count line number and swap first \& last fields:}\\
\entry{35mm}{\{ print \hl{NR, NF, \$1} \}}{use of fields}\\



%%%%%%%%%%%%%%%%%%%%%%%%%%%%%%%%%%%%%%%
\subsection*{Expressions}

\textit{Arithmetic operators:}\\
\begin{tabular}{l l l l l l l}
    +	&
    -	&
    *	&
    /	&
    \%	&
    \textasciitilde	&
    **	\\
\end{tabular}



\textit{Assignment Operators:}\\
\begin{tabular}{l l l l l}
    ++	&
    --	&
    +=	&
    -=	&
    *=	\\
    /=	&
    \%=	&
    \textasciitilde =	&
    **=	&
\end{tabular}



\textit{Relational Operators:}\\
\begin{tabular}{l l l l l l l l}
  < 	&
    >	&
    <=	&
    >=	&
    ==	&
    !=	&
    \textasciitilde	&
    !\textasciitilde	\\
\end{tabular}



\textit{Boolean Operators:}\\
\begin{tabular}{l l l}
    !!	&
    \&\&	&
    !	\\
\end{tabular} \\
\vspace{2mm}
\entry{38mm}{/\textasciicircum\$/ \{ print x += 1 \}}{count lines}\\
\entry{38mm}{NF == 6 \hl{\&\&} NR > 1 \{ \dots \}}{compound expr}\\




%%%%%%%%%%%%%%%%%%%%%%%%%%%%%%%%%%%%%%%
\subsection*{Variables}

\textit{Basic variable  assignment, manipulation:}\\
\entry{35mm}{x = 1}{basic assignment}\\
\entry{35mm}{z = ``Hello''}{strings}\\
\entry{35mm}{y = x + 1}{math}\\
\entry{35mm}{y += 1}{assign operator}\\

\textit{System variables:}\\
{\footnotesize \begin{tabular}{l l}
    ARGC	& \# arguments on command line		\\
    ARGV	& Array $\ni$ cmd line args		\\
    CONVFMT	& String conversion formatter		\\
    ENVIRON	& Array $\ni$ all env vars		\\
    FILENAME	& Current filename			\\
    FNR		& NR relative to current file		\\
    NF		& \# fields in current record		\\
    NR		& \# of the current record		\\
    OFMT	& Output format for numbers		\\
    OFS		& Output field separator		\\
    ORS		& Output record separator		\\
    RLENGTH	& Length of str from match()		\\
    RS		& Record separator			\\
    RSTART	& 1st pos from match()			\\
    SUBSEP	& Separator char for array sub		\\
\end{tabular}}




%%%%%%%%%%%%%%%%%%%%%%%%%%%%%%%%%%%%%%%
\subsection*{Types}

% Strings
\textit{Strings can contain the following escape sequences: }\textbf{\textbackslash a}, \textbf{\textbackslash b}, \textbf{\textbackslash f}, \textbf{\textbackslash n}, \textbf{\textbackslash r}, \textbf{\textbackslash t}, \textbf{\textbackslash v}, \textbf{\textbackslash <oct>}, \textbf{\textbackslash x<hex>}, \textbf{\textbackslash c}\\
% Numerics
% Arrays
\textit{Arrays are associative; iterate/test with for/if-in.}\\



\entry{35mm}{\{ print \hl{str1 ``: ''  str2} \}}{string concat}\\
\entry{35mm}{grade[``blair'']=99}{array assignemnt}\\
\entry{30mm}{split(\$1, name, `` '')}{1st field $\to$ name[]}\\
\entry{35mm}{delete grade[``blair'']}{edit array}\\


%%%%%%%%%%%%%%%%%%%%%%%%%%%%%%%%%%%%%%%
\subsection*{Flow Control}
\textit{Syntax follows C's. Brackets \{ \} are required for multiple (\textbackslash n-separated) lines of statements. }\textbf{break}\textit{ and }\textbf{continue}\textit{ have normal meaning. }\textbf{next}\textit{ and }\textbf{exit}\textit{ goto next input line and exit the script.}\\
% If statement
\code{if(<cond>) \{ <statmt>* \}}\\
% Ternary if
\code{<expr> ? <statemt> : <statemt>}\\
% While
\code{while( <cond> ) \textbackslash n\ \ <statemt>*}\\
% Until
\code{do \textbackslash n <statmt>* \textbackslash n while( <cond> )}\\
% For
\code{for( <statmt>; <test>; <iterate>) \{ \dots \}}\\
% For-in
\code{for( <var> in <array>) \{ <statmt>* \}}\\[1mm]
\textit{Minimum working flow-control examples:}\\
\entry{35mm}{if(x==y) print x}{one-line if}\\
\entry{35mm}{avg > 65 ? ``pass'' : ``fail''}{ternary}\\
\entry{35mm}{while (i <= 4) \{ \dots \}}{line per statement}\\
\entry{35mm}{do \{ \dots \} while (x<=4)}{line per statment}\\
\entry{40mm}{for( i=1; i <= NF; i++) print \dots}{iterate flds}\\
\entry{40mm}{for(itm in myarray) \dots}{iterate array}\\


%%%%%%%%%%%%%%%%%%%%%%%%%%%%%%%%%%%%%%%
\subsection*{Functions}

% Built-in
% Library of functions
\textit{Arithmetic functions:}\\
\begin{tabular}{l l l l l}
    cos		&
    exp		&
    int		&
    log		&
    sin		\\	
    sqrt	&
    atan2	&
    rand	&
    srand	&
\end{tabular}\\
\vspace{1mm} \\
\textit{String functions:}\\
\begin{tabular}{l l l l l}
    gsub	&
    index	&
    length	&
    split   &
    sprintf \\
    sub     &
    substr	&
    tolower	&
    toupper	\\
\end{tabular}
\vspace{1mm} \\
\textit{System functions:}\\
\begin{tabular}{l l l l l}
    getline	&	read from stdin (see sed version)	\\
    system	&	execute system function			\\
    close	&	close pipes or files script opened	\\
    >\quad|	&	output to file or pipe 			\\
\end{tabular} 
\vspace{1mm} \\
\textit{Create and call user-defined functions:}\\
\entry{35mm}{function <str> ( <var>* ) \{}{name, params}\\
\entry{35mm}{\phantom{\quad}<statmt>*}{newline separated}\\
\entry{35mm}{\phantom{\quad}[return <var>]?}{optional return}\\
\entry{35mm}{\}}{}\\
\entry{35mm}{myfunc(\$1, 4, ``mystr'')}{call myfunc}\\



%%%%%%%%%%%%%%%%%%%%%%%%%%%%%%%%%%%%%%%
%\subsection*{System Interaction}

% Pipes
% File I/O
% System calls



%%%%%%%%%%%%%%%%%%%%%%%%%%%%%%%%%%%%%%%
\subsection*{Idioms}
\textit{Parse data given in ``blocks'' (records in lines):}\\
\code{BEGIN \{ FS = ``\textbackslash n''; RS = ``'' \}}\\[1mm]
\textit{Report headers / footers using special sections:}\\
\code{BEGIN \{ print ``Col 1'', ``\textbackslash t'', ``Col 2'' \} }\\
\code{END \{ print ``Total: '', sum \} }\\[1mm]
\textit{Dynamic variable binding by shell:}\\
\code{awk `\{ \dots \} dir=\$(pwd) file1 \dots }\\[1mm]
\textit{Confusing idiom, \$1's mean different things:}\\
\code{shell> cat script}\\
\code{shell> \#! /bin/bash}\\
\code{shell> awk `\$1 == search' search=\$1 file}\\[20mm] %note: kludge!!

%\begin{comment}
%%%%%%%%%%%%%%%%%%%%%%%%%%%%%%%%%%%%%%%
\subsection*{printf Formatting}

\code{printf( <format> [, <argument>]* )}\\[1mm]
% \textit{\dots where }\texttt{-}\textit{ indicates left justification, 1st }\texttt{0}\textit{ a ``zero-fill'' option; }\texttt{pre-decimal point digits a ``minimum width,'' digits following decimal for float-precision, and finally the mandatory }\texttt{<specifier>}\textit{ to indicate a type, $\in$:}\\
\textit{Where }\texttt{<format>}\textit{ is a string containing arbitrary text, escaped special characters, and any number of }\texttt{<FS>}\textit{s (format specifier), each comprised as follows:}\\
\code{<FS> := \%[-]?[0]?[0-9]*(.[0-9]*)?<spec>}\\
\begin{tabular}{l l}
    {[-]}?	&	left justify option		\\
    {[0]}?	&	zero-fill option		\\
    {[0-9]}*	&	minimum print width		\\
    (.[0-9]*)?	&	decimal precision for floats	\\
    <spec>	&	mandatory, see below:		\\
\end{tabular}
\begin{tabular}{l l | l l}
    c	&	\ul{c}haracter		&	
    d	& 	\ul{d}ecimal		\\
    i	& 	\ul{i}nteger		&
    e	&	float			\\
    E	&	float			&
    f	&	float			\\
    g	& 	shortest of \textbf{e} or \textbf{f}	&
    G	&	shortest of \textbf{E} or \textbf{F}	\\
    o	&	unsigned octal		&
    s	& 	\ul{s}tring			\\
    u	&	\ul{u}nsigned decimal	&
    x	&	lowercase he\ul{x}	\\
    X	&	uppercase he\ul{X}	&
    \%	&	literal \%		\\
\end{tabular}
\entry{35mm}{printf(``|\%10s|'')}{| { } { } hello|}\\
\entry{35mm}{printf(``|\%-10s|'')}{|hello { } { } |}\\
\entry{45mm}{printf(``\%d \%s \%d'', 7, ``ate'', 9)}{7 ate 9}\\
\entry{35mm}{printf(``\%09.5f'', $\pi$)}{003.14159}\\
\entry{35mm}{printf(``\%*.*g\textbackslash n'', 5, 3, var)}{dynamic format}\\
%\end{comment}
